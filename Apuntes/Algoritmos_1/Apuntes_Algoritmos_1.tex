% Copyright (c) 2013-02-15 Sosa Juan Cruz
%  Permission is granted to copy, distribute and/or modify this document
%   under the terms of the GNU Free Documentation License, Version 1.3
%   or any later version published by the Free Software Foundation;
%   with no Invariant Sections, no Front-Cover Texts, and no Back-Cover Texts.
%   A copy of the license is included in the section entitled "GNU
%    Free Documentation License". 
\documentclass[a4paper,10pt]{article}
\textheight=25cm %Establece el largo del texto en cada página. El default es 19 cm. 
\textwidth=17cm %Establece el ancho del texto en cada página (en este caso, de 17 cm). El default es 14 cm. 
\topmargin=-1cm %Establece el margen superior. El default es de 3 cm, en este caso la instrucción sube el margen 1 cm hacia arriba. 
\oddsidemargin=0mm %Establece el margen izquierdo de la hoja. El default es de 4.5 cm; sin embargo, con sólo poner esta instrucción el margen queda en 2.5 cm. Si el parámetro es positivo se aumenta este margen y si es negativo disminuye. 
\parindent=0mm %elimina la sangría. 
\usepackage[utf8x]{inputenc} % Paquete de idioma que incluye escritura latina
\usepackage{amssymb}        % Paquete para símbolos matemáticos
\usepackage{amsmath}
\usepackage{graphicx}       % Paquete para insertar imágenes
\usepackage[colorlinks=true,linkcolor=black,urlcolor=black]{hyperref} 
\usepackage{bookmark}       % Índice por Secciones en el PDF
\usepackage[spanish,es-nolists]{babel}
\title{ Apuntes de Algoritmos I }
%\setcounter{secnumdepth}{3}
%\author{Juan Cruz Sosa \\ Guido Rajngewerc \\ Micaela Ranea Sánchez \\ Bruno Winocur}
\date{23 de Mayo de 2013}
%%%%%%%%%%%%%%%% ESTO HACE QUE NO APAREZCA LA NUMERACIÓN EN LAS SECCIONES SALVO EN EL ÍNDICE %%%%%%%%%%%%%%%%%%%%%%
%\makeatletter													 %%
%\renewcommand\@seccntformat[1]{}										 %%
%\makeatother												         %%
%%%%%%%%%%%%%%%%%%%%%%%%%%%%%%%%%%%%%%%
\usepackage{framed}
% Flags varios
\newif\ifversionlarga %Flags para imprimir o nó las demostraciones
\newif\ifcolores %Flags para imprimir los colores o no por si parecen muy gay
\colorestrue
\newcommand{\elemento}[1]{  {\textbf{\underline{#1}:}} \ }
\newcommand{\elemColor}[2] {\elemento {#2}}
%si existe el paquete xcolor entonces las propiedades,operaciones,definiciones,etc son con colores
\IfFileExists{xcolor.sty}
{
	\ifcolores
		\usepackage[usenames,dvipsnames,svgnames,table]{xcolor}
		\renewcommand{\elemColor}[2]{{\color{##1} \elemento {##2}}}
	\fi
} %si no existe el paquete son sin colores
{

}
\newcommand{\propiedadColor}[1]{\elemColor{OliveGreen}{#1}}
\newcommand{\lemaColor}[1]{\elemColor{OliveGreen}{#1}}
\newcommand{\operacionColor}[1]{\elemColor{OliveGreen}{#1}}
\newcommand{\definicionColor}[1]{\elemColor{blue}{#1}}
\newcommand{\observacionColor}[1]{\elemColor{Purple}{#1}}
\newcommand{\teoremaColor}[1]{\elemColor{Mahogany}{#1}}
\newcommand{\notacionColor}[1]{\elemColor{Dandelion}{#1}}
\newcommand{\ejemploColor}[1]{\elemColor{Sepia}{#1}}
% Comandos para algunas expresiones matemáticas específicas %
\providecommand{\abs}[1]{\left\lvert#1\right\rvert}
\providecommand{\norm}[1]{\left\lVert#1\right\rVert}
\newcommand{\cis}[1]{\cos \left( #1 \right) + i \cdot \sin \left( #1 \right)}
% Comandos para las definiciones, observaciones, ejemplos, teoremas, etc.
\newcommand{\definicion}{\definicionColor{Definición}}
\newcommand{\observacion}{\observacionColor{Observación}}
\newcommand{\ejemplo}{\ejemploColor {Ejemplo}}
\newcommand{\teorema}{\teoremaColor {Teorema}}
\newcommand{\propiedad}{\propiedadColor {Propiedad}}
\newcommand{\propiedades}{\propiedadColor{Propiedades}}
\newcommand{\propiedadA}[1]{\propiedadColor{Propiedad (#1)}}
\newcommand{\propiedadesA}[1]{\propiedadColor{Propiedades (#1)}}
\newcommand{\definicionA}[1]{\definicionColor {Definición (#1)}}
\newcommand{\lema}{\lemaColor{Lema}}
\newcommand{\teoremaDe}[1]{\teoremaColor {Teorema de #1}}
\newcommand{\teoremaB}[1]{\teoremaColor {Teorema #1}}
\newcommand{\notacion}{\notacionColor {Notación}}
\newcommand{\nota}{\notacionColor{Nota}}
\newcommand{\puntual}{\elemento {Función de probabilidad puntual}}
\newcommand{\acumulada}{\elemento {Función de distribución acumulada}}
\newcommand{\esperanza}{\elemento {Esperanza}}
\newcommand{\varianza}{\elemento {Varianza}}
\newcommand{\rangoVar}{\elemento {Rango}}
\newcommand{\generadora}{\elemento {Función generadora de momentos}}
\newcommand{\dado}{\mid}
\newcommand{\distribucion}{\elemento {Nombre de la distribución}}
\newcommand{\operaciones}{\operacionColor {Operaciones}}
\newcommand{\operacionesDe}[1]{\operacionColor {Operaciones #1}}
\newcommand{\corolario}{\elemento {Corolario}}
\newcommand{\defsucesion}[2][n]{\left( {#2}_{#1} \right)_{#1 \in \mathbb{N}}}
\newcommand{\deflsucesion}[2]{\forall \varepsilon > 0 \ \ \exists n_0 \ / \ \abs{#1_n-#2} < \varepsilon  \mbox{ si } n \geq n_0}
\newcommand{\rene}[1][n]{\mathbb{R}^{#1}}
\newcommand{\pesc}[2]{\left\langle {#1},{#2} \right\rangle}
% Environments personalizados
%Environment para demostraciones
\newenvironment{demo}
{\begin{framed}\textbf{\emph{Demostración}}: \par\indent}
{\hspace{\stretch{1}}$\blacksquare$\end{framed}}


% Paquetes generales
\usepackage{ifthen}
\usepackage{amssymb}
\usepackage{multicol}
\usepackage[absolute]{textpos}

%%%%%%%%%%%%%%%%%%%%%%%% TWEAKLIST.STY %%%%%%%%%%%%%%%%%%%%%%%%
%% Esto esta copiado de tweaklist.sty, un paquete que encontr� en la
%% web. Define hooks que se ejecutan cada vez que se invoca
%% determinado environment (itemhook, enumhook y descripthook
%%%%%%%%%%%%%%%%%%%%%%%%%%%%%%%%%%%%%%%%%%%%%%%%%%%%%%%%%%%%%%%
\makeatletter
\def\enumhook{}
% \def\enumhooki{}\def\enumhookii{}\def\enumhookiii{}
% \def\enumhookiv{}\def\itemhook{}\def\itemhooki{}\def\itemhookii{}
% \def\itemhookiii{}\def\itemhookiv{}\def\descripthook{}
\def\enumerate{%
  \ifnum \@enumdepth >\thr@@\@toodeep\else
    \advance\@enumdepth\@ne
    \edef\@enumctr{enum\romannumeral\the\@enumdepth}%
      \expandafter
      \list
        \csname label\@enumctr\endcsname
        {\usecounter\@enumctr\def\makelabel##1{\hss\llap{##1}}%
          \enumhook \csname enumhook\romannumeral\the\@enumdepth\endcsname}%
  \fi}
% \def\itemize{%
%   \ifnum \@itemdepth >\thr@@\@toodeep\else
%     \advance\@itemdepth\@ne
%     \edef\@itemitem{labelitem\romannumeral\the\@itemdepth}%
%     \expandafter
%     \list
%       \csname\@itemitem\endcsname
%       {\def\makelabel##1{\hss\llap{##1}}%
%         \itemhook \csname itemhook\romannumeral\the\@itemdepth\endcsname}%
%   \fi}
% \renewenvironment{description}
%                  {\list{}{\labelwidth\z@ \itemindent-\leftmargin
%                           \let\makelabel\descriptionlabel\descripthook}}
%                  {\endlist}
%%%%%%%%%%%%%%%%%%%%%%%% TWEAKLIST.STY %%%%%%%%%%%%%%%%%%%%%%%%


% En las practicas usamos numeros arabigos para los ejercicios.
% Aca cambiamos los enumerate comunes para que usen letras y numeros
% romanos
\newcommand{\arreglarincisos}{%
  \renewcommand{\theenumi}{\alph{enumi}}
  \renewcommand{\theenumii}{\roman{enumii}}
  \renewcommand{\labelenumi}{\theenumi)}
  \renewcommand{\labelenumii}{\theenumii)}
}

%%%%%%%%%%%%%%%%%%%%%%%%%%%%%% PARCIAL %%%%%%%%%%%%%%%%%%%%%%%%
\let\@xa\expandafter
\newcommand{\tituloparcial}{\centerline{\depto -- \lamateria}
  \centerline{\elnombre -- \lafecha}%
  \setlength{\TPHorizModule}{10mm} % Fija las unidades de textpos
  \setlength{\TPVertModule}{\TPHorizModule} % Fija las unidades de
                                % textpos
  \arreglarincisos
  \newcounter{total}% Este contador va a guardar cuantos incisos hay
                    % en el parcial. Si un ejercicio no tiene incisos,
                    % cuenta como un inciso.
  \newcounter{contgrilla} % Para hacer ciclos
  \newcounter{columnainicial} % Se van a usar para los cline cuando un
  \newcounter{columnafinal}   % ejercicio tenga incisos.
  \newcommand{\primerafila}{}
  \newcommand{\segundafila}{}
  \newcommand{\rayitas}{} % Esto va a guardar los \cline de los
                          % ejercicios con incisos, asi queda mas bonito
  \newcommand{\anchodegrilla}{20} % Es para textpos
  \newcommand{\izquierda}{7} % Estos dos le dicen a textpos donde colocar
  \newcommand{\abajo}{2}     % la grilla
  \newcommand{\anchodecasilla}{0.4cm}
  \setcounter{columnainicial}{1}
  \setcounter{total}{0}
  \newcounter{ejercicio}
  \setcounter{ejercicio}{0}
  \newenvironment{ejercicio}[1]
  {%
    \stepcounter{ejercicio}\textbf{Ejercicio \theejercicio. [##1
      puntos]}% Formato
    \renewcommand\@currentlabel{\theejercicio}% Esto es para las
                                % referencias
    \newcommand{\invariante}[2]{%
      {\normalfont\bfseries\ttfamily invariante}%
      \ ####1\hspace{1em}####2%
    }%
    \renewcommand{\problema}[5][result]{
      \encabezadoDeProblema{####1}{####2}{####3}{####4}\hspace{1em}####5}%
  }% Aca se termina el principio del ejercicio
  {% Ahora viene el final
    % Esto suma la cantidad de incisos o 1 si no hubo ninguno
    \ifthenelse{\equal{\value{enumi}}{0}}
    {\addtocounter{total}{1}}
    {\addtocounter{total}{\value{enumi}}}
    \ifthenelse{\equal{\value{ejercicio}}{1}}{}
    {
      \g@addto@macro\primerafila{&} % Si no estoy en el primer ej.
      \g@addto@macro\segundafila{&}
    }
    \ifthenelse{\equal{\value{enumi}}{0}}
    {% No tiene incisos
      \g@addto@macro\primerafila{\multicolumn{1}{|c|}}
      \bgroup% avoid overwriting somebody else's value of \tmp@a
      \protected@edef\tmp@a{\theejercicio}% expand as far as we can
      \@xa\g@addto@macro\@xa\primerafila\@xa{\tmp@a}%
      \egroup% restore old value of \tmp@a, effect of \g@addto.. is
      
      \stepcounter{columnainicial}
    }
    {% Tiene incisos
      % Primero ponemos el encabezado
      \g@addto@macro\primerafila{\multicolumn}% Ahora el numero de items
      \bgroup% avoid overwriting somebody else's value of \tmp@a
      \protected@edef\tmp@a{\arabic{enumi}}% expand as far as we can
      \@xa\g@addto@macro\@xa\primerafila\@xa{\tmp@a}%
      \egroup% restore old value of \tmp@a, effect of \g@addto.. is
      % global 
      % Ahora el formato
      \g@addto@macro\primerafila{{|c|}}%
      % Ahora el numero de ejercicio
      \bgroup% avoid overwriting somebody else's value of \tmp@a
      \protected@edef\tmp@a{\theejercicio}% expand as far as we can
      \@xa\g@addto@macro\@xa\primerafila\@xa{\tmp@a}%
      \egroup% restore old value of \tmp@a, effect of \g@addto.. is
      % global 
      % Ahora armamos la segunda fila
      \g@addto@macro\segundafila{\multicolumn{1}{|c|}{a}}%
      \setcounter{contgrilla}{1}
      \whiledo{\value{contgrilla}<\value{enumi}}
      {%
        \stepcounter{contgrilla}
        \g@addto@macro\segundafila{&\multicolumn{1}{|c|}}
        \bgroup% avoid overwriting somebody else's value of \tmp@a
        \protected@edef\tmp@a{\alph{contgrilla}}% expand as far as we can
        \@xa\g@addto@macro\@xa\segundafila\@xa{\tmp@a}%
        \egroup% restore old value of \tmp@a, effect of \g@addto.. is
        % global 
      }
      % Ahora armo las rayitas
      \setcounter{columnafinal}{\value{columnainicial}}
      \addtocounter{columnafinal}{-1}
      \addtocounter{columnafinal}{\value{enumi}}
      \bgroup% avoid overwriting somebody else's value of \tmp@a
      \protected@edef\tmp@a{\noexpand\cline{%
          \thecolumnainicial-\thecolumnafinal}}%
      \@xa\g@addto@macro\@xa\rayitas\@xa{\tmp@a}%
      \egroup% restore old value of \tmp@a, effect of \g@addto.. is
      \setcounter{columnainicial}{\value{columnafinal}}
      \stepcounter{columnainicial}
    }
    \setcounter{enumi}{0}%
    \vspace{0.2cm}%
  }%
  \newcommand{\tercerafila}{}
  \newcommand{\armartercerafila}{
    \setcounter{contgrilla}{1}
    \whiledo{\value{contgrilla}<\value{total}}
    {\stepcounter{contgrilla}\g@addto@macro\tercerafila{&}}
  }
  \newcommand{\grilla}{%
    \g@addto@macro\primerafila{&\textbf{TOTAL}}
    \g@addto@macro\segundafila{&}
    \g@addto@macro\tercerafila{&}
    \armartercerafila
    \ifthenelse{\equal{\value{total}}{\value{ejercicio}}}
    {% No hubo incisos
      \begin{textblock}{\anchodegrilla}(\izquierda,\abajo)
        \begin{tabular}{|*{\value{total}}{p{\anchodecasilla}|}c|}
          \hline
          \primerafila\\
          \hline
          \tercerafila\\
          \tercerafila\\
          \hline
        \end{tabular}
      \end{textblock}
    }
    {% Hubo incisos
      \begin{textblock}{\anchodegrilla}(\izquierda,\abajo)
        \begin{tabular}{|*{\value{total}}{p{\anchodecasilla}|}c|}
          \hline
          \primerafila\\
          \rayitas
          \segundafila\\
          \hline
          \tercerafila\\
          \tercerafila\\
          \hline
        \end{tabular}
      \end{textblock}
    }
  }%
  \vspace{0.4cm}
  \textbf{LU:}
  
  \textbf{Apellidos:}
  
  \textbf{Nombres:}
  \vspace{0.5cm}
}
%%%%%%%%%%%%%%%%%%%%%%%%%%%%%% PARCIAL %%%%%%%%%%%%%%%%%%%%%%%%

% Esta parte arma cosas que dependen de si estamos usando beamer o no
% tocarEspacios ajusta leftskip y parindent para poder usarlas 
\@ifclassloaded{beamer}{%
  \newcommand{\tocarEspacios}{%
    \addtolength{\leftskip}{4em}%
    \addtolength{\parindent}{-3em}%
  }%
}
{%
  \usepackage[top=1cm,bottom=2cm,left=1cm,right=1cm]{geometry}%
  \usepackage{color}%
  \newcommand{\tocarEspacios}{%
    \addtolength{\leftskip}{2.5em}%
    \addtolength{\parindent}{-3em}%
  }%
}

\@ifundefined{mod}{%
  \newcommand{\mod}{\ \nom{mod}\ }%
}{%
  \renewcommand{\mod}{\ \nom{mod}\ }%
}
% Simbolos varios

% La Z de los numeros enteros
\newcommand{\ent}{\ensuremath{\mathbb{Z}}}
% La R de float
\newcommand{\float}{\ensuremath{\mathbb{R}}}
% El tipo Bool
\newcommand{\bool}{\ensuremath{\mathsf{Bool}}}
\newcommand{\True}{\ensuremath{\mathrm{True}}}
\newcommand{\False}{\ensuremath{\mathrm{False}}}
\newcommand{\Then}{\ensuremath{\rightarrow}}
\newcommand{\Iff}{\ensuremath{\leftrightarrow}}
\newcommand{\implica}{\ensuremath{\longrightarrow}}
\newcommand{\IfThenElse}[3]{\ensuremath{\mathsf{if}\ #1\ \mathsf{then}\ #2\ \mathsf{else}\ #3}}

% Comandos de formato
\newcommand{\nom}[1]{\ensuremath{\mathsf{#1}}}
% Comando para un comentario entre /* */. Font normal
\newcommand{\comentario}[1]{{/*\ #1\ */}}

\newcommand{\ya}{ya especificado en el trabajo anterior}

% Comandos del lenguaje de especificacion
% Selector para sacar elementos de una lista
\newcommand{\selec}{\ensuremath{\leftarrow}}
% La lista vacia
\newcommand{\lv}{[\,]}
% El ++ "bonito"
\newcommand{\masmas}{\ensuremath{+\!\!\!+}}

% Las barritas
\newcommand{\longitud}[1]{\ensuremath{\left|#1\right|}}
\newcommand{\cons}{\nom{cons}}
\newcommand{\indice}{\nom{indice}}
\newcommand{\conc}{\nom{conc}}
\newcommand{\concat}{\nom{concat}}
\newcommand{\cab}{\nom{cab}}
\newcommand{\cola}{\nom{cola}}
\newcommand{\sub}{\nom{sub}}
\newcommand{\en}{\nom{en}}
\newcommand{\cuenta}[2]{\nom{cuenta}\ensuremath{(#1, #2)}}
\newcommand{\suma}{\nom{suma}}
\newcommand{\twodots}{\nom{..}}
\newcommand{\rango}[2]{[#1\twodots#2]}
\newcommand{\rangoac}[2]{(#1\twodots#2]}
\newcommand{\rangoca}[2]{[#1\twodots#2)}
\newcommand{\rangoaa}[2]{(#1\twodots#2)}


% Listas por comprension. El primer parametro es la expresion y el
% segundo tiene los selectores y las condiciones.
\newcommand{\comp}[2]{[\,#1\,|\,#2\,]}
% Listas por extensi�n
\newcommand{\ext}[1]{[\,#1\,]}

% acum: el primer parametro es la expresion, el segundo la definicion
% de la variable de acumulacion, y el tercero los selectores y condiciones.
\newcommand{\acum}[3]{\mathrm{acum}(#1\; | \; #2, #3)}

\newcommand{\sinonimo}[2]{%
  \noindent%
  {\normalfont\bfseries\ttfamily tipo\ }%
  #1\ =\ #2%
  {\normalfont\bfseries\,;\par}
}

\newcommand{\tupla}[2]{\ensuremath{\langle}#1, #2\ensuremath{\rangle}}

% El primer par�metro es el nombre del tipo
% El segundo par�metro es la lista de elementos
\newcommand{\enumerado}[2]{%
  \noindent%
  {\normalfont\bfseries\ttfamily tipo\ }%
  #1\ =\ #2%
  {\normalfont\bfseries\,;\par}
}

\newcommand{\aux}[4]{%
  {
    \addtolength{\leftskip}{1em}
    \addtolength{\parindent}{-2.5em}
    {\normalfont\bfseries\ttfamily aux\ }%
    {\normalfont\ttfamily #1}%
    \ifthenelse{\equal{#2}{}}{}{\ (#2)}\ : #3\, = \ensuremath{#4}%
    {\normalfont\bfseries\,;\par}
  }
}

\newcommand{\encabezadoDeProblema}[4]{%
  % Ponemos la palabrita problema en tt
%  \noindent%
  {\normalfont\bfseries\ttfamily problema}%
  % Ponemos el nombre del problema
  \ %
  {\normalfont\ttfamily #2}%
  \ 
  % Ponemos los parametros
  (#3)%
  \ifthenelse{\equal{#4}{}}{}{%
  \ =\ %
  % Ponemos el nombre del resultado
  {\normalfont\ttfamily #1}%
  % Por ultimo, va el tipo del resultado
  \ : #4}
}

\newcommand{\encabezadoDeTipo}[2]{%
  % Ponemos la palabrita tipo en tt
  {\normalfont\bfseries\ttfamily tipo}%
  % Ponemos el nombre del tipo
  \ %
  {\normalfont\ttfamily #2}%
  \ifthenelse{\equal{#1}{}}{}{$\langle$#1$\rangle$}
}

\newenvironment{problema}[4][result]{%
  % El parametro 1 (opcional) es el nombre del resultado
  % El parametro 2 es el nombre del problema
  % El parametro 3 son los parametros
  % El parametro 4 es el tipo del resultado
  % Preambulo del ambiente problema
  % Tenemos que definir los comandos requiere, asegura, modifica y aux
  \newcommand{\requiere}[2][]{%
    {\normalfont\bfseries\ttfamily requiere}%
    \ifthenelse{\equal{##1}{}}{}{\ {\normalfont\ttfamily ##1} :}\ %
    \ensuremath{##2}%
    {\normalfont\bfseries\,;\par}%
  }
  \newcommand{\asegura}[2][]{%
    {\normalfont\bfseries\ttfamily asegura}%
    \ifthenelse{\equal{##1}{}}{}{\ {\normalfont\ttfamily ##1} :}\
    \ensuremath{##2}%
    {\normalfont\bfseries\,;\par}%
  }
  \newcommand{\modifica}[1]{%
    {\normalfont\bfseries\ttfamily modifica\ }%
    \ensuremath{##1}%
    {\normalfont\bfseries\,;\par}%
  }
  \renewcommand{\aux}[4]{%
    {\normalfont\bfseries\ttfamily aux\ }%
    {\normalfont\ttfamily ##1}%
    \ifthenelse{\equal{##2}{}}{}{\ (##2)}\ : ##3\, = \ensuremath{##4}%
    {\normalfont\bfseries\,;\par}%
  }
  \newcommand{\res}{#1}
  \vspace{1ex}
  \noindent
  \encabezadoDeProblema{#1}{#2}{#3}{#4}
  % Abrimos la llave
  \{\par%
  \tocarEspacios
}
% Ahora viene el cierre del ambiente problema
{
  % Cerramos la llave
  \noindent\}
  \vspace{1ex}
}

\newenvironment{tipo}[2][]{%
  % Preambulo del ambiente tipo
  % Tenemos que definir los comandos observador (con requiere) y aux
  \newcommand{\observador}[3]{%
    {\normalfont\bfseries\ttfamily observador\ }%
    {\normalfont\ttfamily ##1}%
    \ifthenelse{\equal{##2}{}}{}{\ (##2)}\ : ##3%
    {\normalfont\bfseries\,;\par}%
  }
  \newcommand{\requiere}[2][]{{%
    \addtolength{\leftskip}{3em}%
    \setlength{\parindent}{-2em}%
    {\normalfont\bfseries\ttfamily requiere}%
    \ifthenelse{\equal{##1}{}}{}{\ {\normalfont\ttfamily ##1} :}\ 
    \ensuremath{##2}%
    {\normalfont\bfseries\,;\par}}
  }
  \newcommand{\explicacion}[1][]{{%
    \addtolength{\leftskip}{3em}%
    \setlength{\parindent}{-2em}%
    \par \hspace{2.3em} ##1%
    }
  }
  \newcommand{\invariante}[2][]{%
    {\normalfont\bfseries\ttfamily invariante}%
    \ifthenelse{\equal{##1}{}}{}{\ {\normalfont\ttfamily ##1} :}\ 
    \ensuremath{##2}%
    {\normalfont\bfseries\,;\par}%
  }
  \renewcommand{\aux}[4]{%
    {\normalfont\bfseries\ttfamily aux\ }%
    {\normalfont\ttfamily ##1}%
    \ifthenelse{\equal{##2}{}}{}{\ (##2)}\ : ##3\, = \ensuremath{##4}%
    {\normalfont\bfseries\,;\par}%
  }
  \vspace{1ex}
  \noindent
  \encabezadoDeTipo{#1}{#2}
  % Abrimos la llave
  \{\par%
  \tocarEspacios
}
% Ahora viene el cierre del ambiente tipo
{
  % Cerramos la llave
  \noindent\}
  \vspace{1ex}
}

% Cuestiones de enunciados

% Primero definiciones de cosas al estilo title, author, date
\def\materia#1{\gdef\@materia{#1}}
\def\@materia{No especifi\'o la materia}
\def\lamateria{\@materia}

\def\cuatrimestre#1{\gdef\@cuatrimestre{#1}}
\def\@cuatrimestre{No especifi\'o el cuatrimestre}
\def\elcuatrimestre{\@cuatrimestre}

\def\anio#1{\gdef\@anio{#1}}
\def\@anio{No especifi\'o el anio}
\def\elanio{\@anio}

\def\fecha#1{\gdef\@fecha{#1}}
\def\@fecha{\today}
\def\lafecha{\@fecha}

\def\nombre#1{\gdef\@nombre{#1}}
\def\@nombre{No especific'o el nombre}
\def\elnombre{\@nombre}

\def\practica#1{\gdef\@practica{#1}}
\def\@practica{No especifi\'o el n\'umero de pr\'actica}
\def\lapractica{\@practica}

% Esta macro convierte el numero de cuatrimestre a palabras
\newcommand{\cuatrimestreLindo}{
  \ifthenelse{\equal{\elcuatrimestre}{1}}
  {Primer cuatrimestre}
  {\ifthenelse{\equal{\elcuatrimestre}{2}}
  {Segundo cuatrimestre}
  {Verano}}
}

\newcommand{\depto}{{UBA -- Facultad de Ciencias Exactas y Naturales --
      Departamento de Computaci\'on}}

\newcommand{\titulopractica}{
  \centerline{\depto}
  \vspace{1ex}
  \centerline{{\Large\lamateria}}
  \vspace{0.5ex}
  \centerline{\cuatrimestreLindo de \elanio}
  \vspace{2ex}
  \centerline{{\huge Pr\'actica \lapractica -- \elnombre}}
  \vspace{5ex}
  \arreglarincisos
  \newcounter{ejercicio}
  \newenvironment{ejercicio}{\stepcounter{ejercicio}\textbf{Ejercicio
      \theejercicio}%
    \renewcommand\@currentlabel{\theejercicio}%
  }{\vspace{0.2cm}}
}  

\newcommand{\titulotp}{
  \centerline{\depto}
  \vspace{1ex}
  \centerline{{\Large\lamateria}}
  \vspace{0.5ex}
  \centerline{\cuatrimestreLindo de \elanio}
  \vspace{0.5ex}
  \centerline{\lafecha}
  \vspace{2ex}
  \centerline{{\huge\elnombre}}
  \vspace{5ex}
}

% AMBIENTE CONSIGNAS
% Se usa en el TP para ir agregando las cosas que tienen que resolver
% los alumnos.
% Dentro del ambiente hay que usar \item para cada consigna

\newcounter{consigna}
\setcounter{consigna}{0}

\newenvironment{consignas}{%
  \newcommand{\consigna}{\stepcounter{consigna}\textbf{\theconsigna.}}%
  \newcommand{\ejercicio}[1]{\item ##1 }
  \renewcommand{\problema}[5][result]{\item
    \encabezadoDeProblema{##1}{##2}{##3}{##4}\hspace{1em}##5}%
  \newcommand{\invariante}[2]{\item%
    {\normalfont\bfseries\ttfamily invariante}%
    \ ##1\hspace{1em}##2%
  }
  \renewcommand{\aux}[4]{\item%
    {\normalfont\bfseries\ttfamily aux\ }%
    {\normalfont\ttfamily ##1}%
    \ifthenelse{\equal{##2}{}}{}{\ (##2)}\ : ##3 \hspace{1em}##4%
  }
  % Comienza la lista de consignas
  \begin{list}{\consigna}{%
      \setlength{\itemsep}{0.5em}%
      \setlength{\parsep}{0cm}%
    }
}%
{\end{list}}


% MACROS ESPECIFICAS DE IMPERATIVO

% El primer parametro es el nombre del segundo parametro del ==
% El segundo parametro es el nombre del tipo
\newcommand{\eligualigual}[2]%
{%
\begin{problema}{operator==}{this,#1 : #2}{Bool}
  \asegura{result == (this == #1)}
\end{problema}
}

% Manejo de listas
% LaTeX deja demasiado espacio entre los items para nuestros prop�sitos.
\renewcommand{\enumhook}{\setlength{\itemsep}{-4pt}}

% Esto ajusta el espacio que se deja antes y despu�s de los
% environments multicol
\setlength{\multicolsep}{5pt}

\makeatother

\begin{document}
\maketitle
\tableofcontents 
\clearpage
\section{Especificación}
\subsection{Introducción}
\begin{itemize}
	\item ¿Qué es una computadora? \\
	Es un dispositivo para hacer operaciones aritméticas calculadas en base a algoritmos
	\item ¿Que es un algoritmo? \\
	Un algoritmo es una sucesión finita de pasos primitivos para resolver un problema
	\item Etapas en el desarrollo de programas 
	\begin{itemize}
		\item Especificación: Definición clara y precisa del problema
		\item Diseño: Elegir una solución y dividir el problema en partes
		\item Programación: Escribir algoritmos y especificarlos en un lenguaje
		\item Validación: Ver si el programa cumple con lo especificado
		\item Mantenimiento: Corregir errores y adaptarlo a nuevos requerimientos
	\end{itemize}
\end{itemize}
\subsection{Etapa de especificación}
\begin{itemize}
	\item Definición de especificación \\
	Una especificación es un contrato, un lenguaje, para formalizar una descripción del problema. La especificación de un programa consta de
	\begin{itemize}
		\item Nombre: Da nombre al programa
		\item Argumentos E/S: Describe que recibe como parámetros y que devuelve
		\item Precondición: Dice cuales entradas son aceptables (requiere)
		\item Poscondición: Dice que propiedades cumple la salida (asegura)
	\end{itemize}
	\item Contrato de la especificación \\
	El programador hace un programa de acuerdo a una especificación. El usuario suministra datos que no violan la precondición. Luego el programa devuelve la salida sin violar la poscondición.
	El programa viola la especificación cuando
	\begin{itemize}
		\item Transforma datos que cumplen la precondición en datos que no cumplen la poscondición
	\end{itemize}
	El programa no viola la especificación cuando
	\begin{itemize}
		\item Se cuelga con datos que no cumplen la precondición
		\item Transforma datos que no cumplen la precondición en datos que cumplen la poscondición
		\item Transforma datos que no cumplen la precondición en datos que no cumplen la poscondición
	\end{itemize}
	\item Sobre-especificación y Sub-especificación
	\begin{itemize}
		\item Sobre-especificación: Imponer una poscondición más restrictiva de lo que se necesita o una precondición más laxa
		\item Sub-especificación: Imponer una poscondición más laxa o una precondición más restrictiva de lo que se necesita
	\end{itemize}
\end{itemize}
\subsection{Etapa de diseño}
\begin{itemize}
	\item Se divide en partes el problema y se decide que parte resuelve que cosa
	\item Se decide si es necesario dividir las distintas partes en diferentes máquinas
	\item Se especifica cada parte por separado
	\item Un programador recibe una sola parte a la vez
\end{itemize}
\subsection{Etapa de programación}
\begin{itemize}
	\item Se escribe un algoritmo para resolver el problema que buscar llegar a la respuesta en una cantidad finita de pasos
	\item Se traduce el algoritmo a un lenguaje que la computadora entienda
	\item Cada lenguaje tiene su propósito como adecuarse a ciertas tareas, etc.
\end{itemize}
\subsection{Validación}
	La validación es para asegurarse de que el programa cumple la especificación del problema
	Hay dos técnicas
	\begin{itemize}
		\item Testing
		\begin{itemize}
			\item Probar el programa con muchos datos y controlar si cumple la especificación (cuantos más mas seguro es el testing)
			\item Se detectan con suerte los errores encontrados
			\item No es infalible
		\end{itemize}
		\item Verificación formal (Demostración formal)
		\begin{itemize}
			\item Es demostrar matemáticamente que un programa cumple con la especificación
			\item Requiere que el lenguaje de especificación sea formal y admita teoría de prueba
			\item La demostración cubre los infinitos valores de entrada que no cumple el testing
			\item Es infalible
		\end{itemize}
	\end{itemize}
\subsection{Lenguajes formales}
	\begin{itemize}
		\item Sintaxis sencilla
		\item Limitan lo que se puede expresar
		\item Explicitan suposiciones
		\item Relacionan formalmente lo escrito (sintaxis) con el significado (semántica)
		\item Tienen cálculo para transformar expresiones válidas a otras también válidas
	\end{itemize}
\subsection{Lógica proposicional}
	\begin{itemize}
		\item Símbolos: $true,false,\perp,\neg,\wedge,\vee,\rightarrow,\leftrightarrow,(,)$
		\item Variables proposicionales: $p,q,r,\dots$
		\item Fórmula: 
		\begin{itemize}	
			\item $true,false,\perp$ son fórmula
			\item Cada variable proposicional son fórmula
			\item $A$ es fórmula, entonces $\neg A$ es fórmula
			\item $A_1,\dots,A_n$ son fórmula,entonces $(A_1 \wedge \dots \wedge A_n)$ son fórmula
			\item $A_1,\dots,A_n$ son fórmula,entonces $(A_1 \vee \dots \vee A_n)$ son fórmula
			\item $A$ y $B$ fórmula, entonces $(A \rightarrow B)$ es fórmula
			\item $A$ y $B$ fórmula, entonces $(A \leftrightarrow B)$ es fórmula
		\end{itemize}
		\item Semántica clásica
		\begin{itemize}
			\item 2 valores de verdad
			\begin{itemize}
				\item verdadero (1) ($true$)
				\item falso (0) ($false$)
			\end{itemize}
			\item Interpretación
			\begin{itemize}
				\item $\neg \equiv$ "no" (negación)
				\item $\wedge \equiv$ "y" (conjunción)
				\item $\vee \equiv$ "o" (disyunción, no exclusivo)
				\item $\rightarrow \equiv$ "si...entonces" (implicación)
				\item $\leftrightarrow \equiv$ "si y solo si" (equivalencia,doble implicación) 
			\end{itemize}
			\item Tablas de verdad
			\begin{center}
				\begin{tabular}{| l | c | }
     				\hline
     				$p$ & $\neg p$  \\ \hline
     				1 & 0  \\ \hline
     				0 & 1  \\ 
     				\hline
   				\end{tabular}
   				\begin{tabular}{| l | c | c | }
     				\hline
     				$p$ & $q$ & $p \wedge q$ \\ \hline
     				0 & 0 & 0 \\ \hline
     				0 & 1 & 0 \\ \hline
     				1 & 0 & 0 \\ \hline
     				1 & 1 & 1 \\ 
     				\hline
   				\end{tabular}
   				\begin{tabular}{| l | c | c | }
     				\hline
     				$p$ & $q$ & $p \vee q$ \\ \hline
     				0 & 0 & 0 \\ \hline
     				0 & 1 & 1 \\ \hline
     				1 & 0 & 1 \\ \hline
     				1 & 1 & 1 \\ 
     				\hline
   				\end{tabular}
   				\begin{tabular}{| l | c | c | }
     				\hline
     				$p$ & $q$ & $p \rightarrow q$ \\ \hline
     				0 & 0 & 1 \\ \hline
     				0 & 1 & 1 \\ \hline
     				1 & 0 & 0 \\ \hline
     				1 & 1 & 1 \\ 
     				\hline
   				\end{tabular}
   				\begin{tabular}{| l | c | c | }
     				\hline
     				$p$ & $q$ & $p \leftrightarrow q$ \\ \hline
     				0 & 0 & 1 \\ \hline
     				0 & 1 & 0 \\ \hline
     				1 & 0 & 0 \\ \hline
     				1 & 1 & 1 \\ 
     				\hline
   				\end{tabular}
 			\end{center}
		\end{itemize}
		\item Semántica trivaluada
		\begin{itemize}
			\item 4 valores de verdad
			\begin{itemize}
				\item verdadero (1) ($true$)
				\item falso (0) ($false$)
				\item indefinido ($-$)
			\end{itemize}
			\item Interpretación
			\begin{itemize}
				\item $\neg \equiv$ "no" (negación)
				\item $\wedge \equiv$ "y" (conjunción)
				\item $\vee \equiv$ "o" (disyunción, no exclusivo)
				\item $\rightarrow \equiv$ "si...entonces" (implicación)
				\item $\leftrightarrow \equiv$ "si y solo si" (equivalencia,doble implicación) 
			\end{itemize}
			\item Secuencial: se lee de izquierda a derecha. La evaluación termina cuando se puede deducir el valor de verdad aunque lo otro esté indefinido
			\item Tablas de verdad
			\begin{center}
				\begin{tabular}{| l | c | }
     				\hline
     				$p$ & $\neg p$  \\ \hline
     				1 & 0  \\ \hline
     				0 & 1  \\ \hline
     				$-$ & $-$ \\
     				\hline
   				\end{tabular}
   				\begin{tabular}{| l | c | c | }
     				\hline
     				$p$ & $q$ & $p \wedge q$ \\ \hline
     				0 & 0 & 0 \\ \hline
     				0 & 1 & 0 \\ \hline
     				1 & 0 & 0 \\ \hline
     				1 & 1 & 1 \\ \hline
     				0 & $-$ & 0 \\ \hline
     				1 & $-$ & $-$ \\ \hline
     				$-$ & 0 & $-$ \\ \hline
     				$-$ & 1 & $-$ \\ 
     				\hline
   				\end{tabular}
   				\begin{tabular}{| l | c | c | }
     				\hline
     				$p$ & $q$ & $p \vee q$ \\ \hline
     				0 & 0 & 0 \\ \hline
     				0 & 1 & 1 \\ \hline
     				1 & 0 & 1 \\ \hline
     				1 & 1 & 1 \\ \hline
     				0 & $-$ & $-$ \\ \hline
     				1 & $-$ & 1 \\ \hline
     				$-$ & 0 & $-$ \\ \hline
     				$-$ & 1 & $-$ \\ 
     				\hline
   				\end{tabular}
   				\begin{tabular}{| l | c | c | }
     				\hline
     				$p$ & $q$ & $p \rightarrow q$ \\ \hline
     				0 & 0 & 1 \\ \hline
     				0 & 1 & 1 \\ \hline
     				1 & 0 & 0 \\ \hline
     				1 & 1 & 1 \\ \hline
     				0 & $-$ & 1 \\ \hline
     				1 & $-$ & $-$ \\ \hline
     				$-$ & 0 & $-$ \\ \hline
     				$-$ & 1 & $-$ \\ 
     				\hline
   				\end{tabular}
   				\begin{tabular}{| l | c | c | }
     				\hline
     				$p$ & $q$ & $p \leftrightarrow q$ \\ \hline
     				0 & 0 & 1 \\ \hline
     				0 & 1 & 0 \\ \hline
     				1 & 0 & 0 \\ \hline
     				1 & 1 & 1 \\ \hline
     				0 & $-$ & $-$ \\ \hline
     				1 & $-$ & $-$ \\ \hline
     				$-$ & 0 & $-$ \\ \hline
     				$-$ & 1 & $-$ \\ 
     				\hline
   				\end{tabular}
 			\end{center}
 			\item Tautología: Una fórmula es tautología si toma el valor 1 ($true$) para todos los posibles valores de sus variables proposicionales
 			\item Contradicción: Una fórmula es tautología si toma el valor 0 ($false$) para todos los posibles valores de sus variables proposicionales
 			\item Contingencia: Cuando no es ni tautología ni contradicción
 			\item Relación de fuerza: $A$ es más fuerte que $B$ cuando $A \rightarrow B$ es tautología (A fuerza a B o B es más débil que A)
		\end{itemize}
	\end{itemize}
\subsection{Tipos de datos}
	\begin{itemize}
		\item Definición: Conjunto de valores con operaciones (Todos los tipos tienen $\perp$)
		\item Tipo Int
		\begin{itemize}
			\item $\in \mathbb{Z}$
			\item Constantes: $0;1;-1;2;-2;\dots$
			\item Operaciones aritméticas
			\begin{itemize}
				\item $a + b$ (suma)
				\item $a - b$ (resta)
				\item $a * b$ (multiplicación)
				\item $a \mbox{ div } b$ (división entera)
				\item $a \mbox{ mod } b$ (resto)
				\item $a^{b}$ o $pot(a,b)$ (potenciación)
				\item $abs(a)$ (valor absoluto)
			\end{itemize}
			\item Comparaciones
			\begin{itemize}
				\item $a == b$ (igual)
				\item $a \neq b$ (distinto)
				\item $a < b$ (menor)
				\item $a \leq b$ (menor o igual)
				\item $a > b$ (mayor)
				\item $a \geq b$ (mayor o igual)
			\end{itemize}
		\end{itemize}
		\item Tipo Bool
		\begin{itemize}
			\item $\in \left\lbrace 0, 1 \right\rbrace$
			\item Constantes: $true,false, \perp$
			\item Operaciones aritméticas
			\begin{itemize}
				\item $no(a)$ ($\neg A =$ $no(A)$)
				\item $a$ \&\& $b$ ($A \wedge B = A$ \&\& $B$)
				\item $a \mid \mid b$ ($A \vee B = A \mid \mid B$)
				\item $a$ $\rightarrow$ $b$ 
				\item $a$ $\leftrightarrow$ $b$ 
			\end{itemize}
			\item Comparaciones
			\begin{itemize}
				\item $a == b$ (igual)
				\item $a \neq b$ (distinto)
			\end{itemize}
		\end{itemize}
		\item Tipo Float
		\begin{itemize}
			\item $\in \mathbb{R}$
			\item Constantes: $0;1;-1;2;-2;\pi,e,\dots$
			\item Operaciones
			\begin{itemize}
				\item $a + b$ (suma)
				\item $a - b$ (resta)
				\item $a \cdot b$ (multiplicación)
				\item $a / b$ (división real)
				\item $a^{b}$ o $pot(a,b)$ (potenciación)
				\item $abs(a)$ (valor absoluto)
				\item $\log_{b}(a)$ (logarítmo)
				\item $sen(a),cos(a),tan(a),asen(a),\dots$
			\end{itemize}
			\item Comparaciones
			\begin{itemize}
				\item $a == b$ (igual)
				\item $a \neq b$ (distinto)
				\item $a < b$ (menor)
				\item $a \leq b$ (menor o igual)
				\item $a > b$ (mayor)
				\item $a \geq b$ (mayor o igual)
			\end{itemize}
		\end{itemize}
		\item Tipo Char
		\begin{itemize}
			\item $\in$ Alfabeto
			\item Constantes: 'a',\dots,'z','A',\dots,'Z','0',\dots,'9','á',\dots,'ú'
			\item Funciones
			\begin{itemize}
				\item $ord(a)$ (numera los caracteres)
				\item $char(a)$ (de el número da el carácter)
			\end{itemize}
			\item Comparaciones
			\begin{itemize}
				\item $a == b$ ($ord(a) = ord(b)$)
				\item $a \neq b$ ($ord(a) \neq ord(b)$)
				\item $a < b$ ($ord(a) < ord(b)$)
				\item $a \leq b$ ($ord(a) \leq ord(b)$)
				\item $a > b$ ($ord(a) > ord(b)$)
				\item $a \geq b$ ($ord(a) \geq ord(b)$)
			\end{itemize}
		\end{itemize}
	\end{itemize}
\subsection{Funciones auxiliares}
	\begin{itemize}
		\item Las funciones auxiliares son una expresión del lenguaje que hace reemplazos sintácticos
		\item No se pueden usar problemas para resolver problemas pero si usar funciones auxiliares para resolverlo
		\item Permiten modularizar el problema
		\item Facilitan la lectura
	\end{itemize}
\subsection{Tipos compuestos}
	\begin{itemize}
		\item Cada valor de tipo básico representa un elemento atómico e indivisible
		\begin{itemize}
			\item Int
			\item Float
			\item Bool 
			\item Char
		\end{itemize}
		\item Un valor de tipo compuesto puede contener información que puede ser dividida en componente de otro tipo
		\item Los observadores son funciones que se aplican a valores del tipo compuesto y devuelve el valor de sus componentes
		\item Los observadores definen al tipo compuesto (si todos los observadores de dos instancias del mismo tipo dan iguales entonces son iguales)
		\item Los observadores pueden tener precondición (si vale la precondición el observador no se indefine)
		\item A las condiciones o reestricciones que tenga el tipo son invariantes del tipo
		\item Los problemas que reciben argumentos de un tipo compuesto automáticamente agregan a la precondición el invariante del tipo
	\end{itemize}
\section{Programación Funcional}
\subsection{Introducción}
	\begin{itemize}
		\item Paradigma: Definición del modo en que se especifica el cómputo
		\item Todo lenguaje pertenece a un paradigma
		\item Paradigmas: 		
		\begin{itemize}
			\item Imperativo (C,Basic,Ada,Clu)
			\item en Objetos (Smaltalk)
			\item orientado a objetos (C++,C\#,Java)
			\item funcional (Lisp,Gopher,Haskell)
			\item lógica (Prolog)
		\end{itemize}
		\item Un programa en funcional es una colección en funciones que transforma datos de entrada en datos de salida
		\item Una expresión es una tira de símbolos que denotan un valor
		\item Hay expresiones
		\begin{itemize}
			\item Atómicas (formas normales) son las que representan un valor
			\item Compuestas que son combinaciones de expresiones atómicas y operaciones
		\end{itemize}
		\item Un programa en lenguaje funcional es un conjunto de ecuaciones que definen funciones
	\end{itemize}
\subsection{Recursividad}
	\begin{itemize}
		\item Una función recursiva consta de 
		\begin{itemize}
			\item Un caso base (para poder terminar)
			\item Una llamada recursiva (que se acerque al caso base)
		\end{itemize}
		\item Las funciones recursivas vendrían a reemplazar los ciclos en lenguajes imperativos
		\item Para definir funciones recursivas se usan ecuaciones orientadas donde
		\begin{itemize}
			\item El lado izquierdo es la expresión a definir
			\item El lado derecho es la definición
		\end{itemize}
		\item Para el cálculo de un valor de una expresión que haga una llamada a una función se hace el reemplazo sintáctico de lo de la izquierda por lo de la derecha
	\end{itemize}
\subsection{Características de Haskell}
	\begin{itemize}
		\item Transparencia referencial: Propiedad del lenguaje que garantiza que el valor de una expresión depende exclusivamente de sus subexpresiones
		\item Cada expresión es unívoca, representa siempre el mismo valor, a diferencia de otros paradigmas en que el significado de una expresión depende del contexto
		\item Reducción: Proceso para reemplazar una subexpresión por su definición, donde la expresión tiene reducida está más cerca de una forma normal (aunque puede ser más larga que la original)
		\item Todo valor es un tipo de dato (inclusive funciones)
		\item Toda expresión bien formada representa un valor
		\item Fuertemente tipado: Todos los tipos son explicitos, no se le puede pasar a una función un tipo de dato que no espera)
		\item Estáticamente tipado: Se definen todos los tipos antes de correr el programa y no pueden cambiar su tipo
		\item Polifórmico: La función puede definirse sobre un tipo genérico (cualquier tipo)
		\item Currificado: Evita el uso de signos de puntuación tales como comas y paréntesis  
	\end{itemize}
\subsection{Tipos de datos}
    \begin{itemize}
		\item Tipos de datos abstractos
	    	\begin{itemize}
			\item No se conoce como se representan los valores de de un tipo de dato (Su representación está encapsulada y oculta)
			\item Solo se conocen sus operaciones
			\item Ejemplo: Int, Float
	    	\end{itemize}
		\item Tipos de datos algebraicos
	    	\begin{itemize}
			\item Se conoce la forma de cada elemento
			\item Para crear un tipo algebraico se escribe la forma que tiene cada elemento
			\item La clausula de la definición consta de
			\begin{itemize}
		   		\item La palabra data
		    		\item Constructores
		    		\begin{itemize}
					\item Nombre de la constante
					\item Argumentos:  No tienen reglas de inferencia y forman expresiones atómicas
		    		\end{itemize}
		    		\item Ejemplo
		    		\begin{itemize}
		    			\item $data$ Bool = $True$ $|$ $False$
		    			\item $data$ Figura = Circ $Float$ $|$ Rect $Float$ $Floati$
				\end{itemize}	
		    	\end{itemize}
			\item Pattern matching
		    	\begin{itemize}
				\item Es el mecanismo para ver como está construido un elemento de tipo algebraico
				\item Patterns: Expresiones del lenguaje tomadas solamente por constructores y variables que no se repiten
				\item Matching: Decide si la expresión coincide por su forma con el patrón (si existe). Entonces liga las variables del patrón a las subexpresiones correspondientes
			\end{itemize}
			\item Tipos algebráicos recursivos: Son tipos algebraicos que toman como argumento de algún constructor un tipo del mismo que se está definiendo. Ej: $data$ Nat = Z $|$ S $Nat$
			\item Recursión estructural: Es lo que permite definir funciones (usando \textit{pattern matching}). Se hace recursión sobre las estructuras de los datos.
			\item Sinónimo de tipos: Aliasing de tipos
		\end{itemize}
	\end{itemize}
\subsection{Inducción estructural}
    \begin{itemize}
	\item Orden bien fundado
	\begin{itemize}
	    \item Solo se puede aplicar inducción sobre un orden bien fundado
	    \item Orden parcial
	    \begin{itemize}
		\item Antisimétrico
		\item Reflexivo
		\item Transitivo
	    \end{itemize}
	    \item Orden total
	    \begin{itemize}
		\item Todo los elementos son comparables ($\forall a,b \in A, a \guilsinglleft b$ o $b \guilsinglleft a$)
	    \end{itemize}
	    \item Orden bien fundado
	    \begin{itemize}
		\item $\guilsinglleft$ es un orden bien fundado sobre A si $\guilsinglleft$ es un orden total y cada subconjunto no vacío tiene un mínimo
	    \end{itemize}
	\end{itemize}
	\item Inducción estructural: Sea $P(X)$ una prop de un elemento de un tipo algebráico T. Queremos probar que vale $P(X) \; \forall X:T$ 
	\begin{itemize}
	    \item Axioma de inducción: Sea $c:T \longrightarrow \mathbb{N}$. Si 
	    \begin{itemize}
		\item $P(y) \; \forall y:T$ tq $c(y)=0$
		\item $(c(y) = n \Rightarrow P(y)) \; \forall y:T \Rightarrow (c(y) = n+1 \Rightarrow P(y)) \; \forall y:T$
	    \end{itemize}
	    \item Entonces $P(y) \; \forall y:T$
	    \item Axioma de inducción completa:
	    \begin{itemize}
	   	\item $P(y) \; \forall y:T$ tq $c(y)=0$
		\item $\forall y:T \; (c(y) \leq n \Rightarrow P(y)) \; \Rightarrow \; \forall y:T (c(y)=n+1 \Rightarrow P(y))$
	    \end{itemize}
	\end{itemize}
	\item Función variante
	\begin{itemize}
	   \item Definición: $f$::$T_1 \longrightarrow \dots \longrightarrow T_n$ función.Se le llama función variante de $f$ a una función $F_{V_f} : T_1 \times \dots \times T_{n-1} \longrightarrow \mathbb{Z}$ estrictamente decreciente en las llamadas. Es decir para cada ecuación $f: \overline{x} = \dots f \overline{y_1} \dots f \overline{y_m}$ se cumple $$F_{V_f}(\overline{x}) > F_{V_f}(\overline{y_i}) \mbox{ para cada } i:1 \dots n$$
	   \item Teorema (Condición suficiente de terminación): Sea $f$::$T_1 \longrightarrow \dots \longrightarrow T_n$ y $F_{V_f} : T_1 \times \dots \times T_n \longrightarrow \mathbb{Z}$ una función variante de f
	   	\subitem Si $F_{V_f}(\overline{x}) \leq 0 \Rightarrow$ $f$ se define mediante una ecuación de caso base.
		Entonces $f$ termina, para cada $\overline{x}$ en $T_1 \times \dots \times T_{n-1}$
	    \item Corolario: $\forall \overline{x} \; T_1 \times \dots \times T_{n-1}$, $f \overline{x}$ termina en a lo sumo $F_{V_f}(\overline{x})$ llamadas recursivas
	\end{itemize}
    \end{itemize}
\subsection{Reducción}
    \begin{itemize}
	\item Modelo de computo: Especifación que define como se calcula el valor de una expresión
	\item Reducción:
	\begin{itemize}
	    \item Lado izquierdo: redex (Reducible expression) o radical
	    \item Lado derecho: subexpresión reemplazante
	\end{itemize}
	\item Las expresiones se reducen hasta llegar una forma normal
	\item Formas normales: Constantes y constructores
	\item Mecanismo de reducción: 
	\begin{itemize}
	    \item Si la expresión está en una forma normal, ya está
	    \item Si no busca un redex
	    \item Reemplaza el redex y vuelve a empezar
	\end{itemize}
	\item Las expresiones que no terminan en una forma normal se llama normal se llaman indefinidas (Se representan con $\perp$,bottom, si se evalua bottom se indefine)
	\item Funciones
	\begin{itemize}
	    \item parcial: $f x$ puede devolver $\perp$
	    \item total: $f x$ puede devolver $\perp$
	    \item estricta: $f \perp \leadsto \perp$
	    \item no estricta $f \perp \leadsto$ valor
	\end{itemize}
	\item Ordenes:
	\begin{itemize}
	    \item Aplicativo
	    \begin{itemize}
		\item Primero redex internos
		\item Primero argumentos, después la aplicación
		\item Todas las funciones son estrictas
	    \end{itemize}
	    \item Normal
	    \begin{itemize}
		\item El redex más externo para hacer pattern matching
		\item Primero la aplicación, después los argumentos (si se necesitan)
		\item Hay estrictas y no estrictas, depende si necesiten evaluar todos sus argumentos
	    \end{itemize}
	    \item Ambos leen de izquierda a derecha
	    \item El orden normal siempre encuentra la forma normal
	\end{itemize}
	\item Evaluación lazy: es el orden de evaluación de haskell, es normal, pero aprovechando la transparencia referencial
	\item Alto orden
	\begin{itemize}
	    \item Las funciones de alto orden son aquellas que reciben o devuelven funciones
	    \item Iteran una función arbitraria sobre una estructura de datos, en un cierto orden
	    \item Construyen un valor de retorno
	    \item Ejemplos:
	    \begin{itemize}
		\item map: Aplica la función que le pasas como argumento de entrada a cada unos de los elementos de la listá pasada por argumento
		\item fiter: Filtra los elementos de la lista pasada por argumento, con los elementos que cumplen con una condición, dada por la función pasada como parámetro de entrada
		\item fold: Reducen la lista pasada como argumento a un solo valor usando la función pasada como parámetro de entrada
	    \end{itemize} 
	\end{itemize}
    \end{itemize}
\section{Imperativo}
\subsection{Introducción}
    \begin{itemize}
    	\item Modelo de cómputo: Un programa consta de instrucciones que especifican como operar con los datos, que están alojados en memoria
	\item Variables:
	\begin{itemize}
	    \item Corresponden a posiciones de memoria RAM
	    \item Se mantienen a lo largo del programa mientras no se cambie explícitamente su valor
	    \item Por todo esto se pierde la transparencia referencial
	    \item Toda variable tiene un tipo asociado y se declaran dando su tipo y su nombre
	\end{itemize} 
	\item Operaciones fundamentales:
	\begin{itemize}
	    \item Asignación: modifica el valor de una variable evaluando la expresión de la derecha y se obtiene un valor. El valor obtenido se copia en la memoria (dondé está la variable).El resto de la memoria no cambia.\textbf{Sintaxis}: (a = b;)
	    \item Control de flujo: 
	    \begin{itemize}
		\item Return: Termina la ejecución de una función, retornando el control a su invocación y devolviendo el valor de la expresión como resultado.\textbf{Sintaxis}: return $a$;
		\item Condicional: Es una operación que dada una condición especifica que bloque de instrucciones ejecuta el programa si se cumple la condición y que bloque si nó.\textbf{Sintaxis}: if(B) uno else dos
		\subitem donde B es la condición (booleana), uno y dos los bloques que se ejecutan si se cumple B o no respectivamente.
		\item Ciclos: Es una operación que dada una condición ejecuta un bloque de instrucciones tantas veces (0 o más veces) mientras se cumpla la condición. \textbf{Sintaxis}: while(B) cuerpo
		\subitem donde B es la condición booleana sin efectos colaterales
	    \end{itemize}
	\end{itemize}
	\item Funciones
	\begin{itemize}
	    \item Las funciones pueden devolver más de un valor
	    \item Hay dos formas de pasar argumentos (referencia y copia)
	    \item Las funciones no son valores y no hay mecanismos para realizar operaciones entre funciones
	    \item Los argumentos de entrada de una función se comportan como variables
	    \subitem Los valores se toman cuando se invoca a la función
	    \subitem Se puede modificar los argumentos de entrada y el valor modificado no guarda relación con las variables de quien real
	\end{itemize}
	\item Programa en C++: Colección de definiciones de tipos y funciones
	\item Semántica de la programación imperativa
	\begin{itemize}
	    \item Conjunto de variables con los datos del problema
	    \item Se ejecuta una serie de instrucciones que van cambiando los valores de las variables
	    \item Cuando el programa termina los valores finales de las variables deberían tener la respuesta
	\end{itemize}
	\item Pasajes de argumentos a la función
	\begin{itemize}
	    \item Por copia: Coloca en la posición de memoria del argumento el valor de una expresión usada en la invocación.
	    \item Por referencia: 
	    \begin{itemize}
		\item La función recibe una posición de memoria donde encontrar el argumento
		\item La función puede leer y escribir el valor de la memoria 
	    \end{itemize}
	\end{itemize}
	\item Referencias: Hace referencia a una variable (actúa como aliasing de variables)
    \end{itemize}
\subsection{Demostraciones de correctitud}
    \begin{itemize}
    	\item Transformaciones de Estados
	\begin{itemize}
	    \item Se llama transformación de estado a los valores de todas sus variables en un punto de ejecución
	    \item Un programa es una sucesión de estados
	    \item Se agregan a la especificación
	    \begin{itemize}
		\item vale $P$
		\item vale etiqueta: $P$
		\item estado nombre (De un nombre a un punto del programa)
		\item pre($nombre\_variable$) (Se pone dentro de las funciones para indicar el valor de los argumentos de la función)
		\item local($nombre\_variable$) si el programa modifica los argumentos de entrada de esa variable
		\item implica: son afirmaciones (del estado de un programa) que se desprenden de las afirmaciones anteriores
	    \end{itemize}
	\end{itemize}
	\item Demostración if
	\begin{itemize}
	    \item \textbf{if}($B$) \textbf{then} uno \textbf{else} dos
	    \item Componentes:
	    \begin{itemize}
		\item Guarda (B)
		\item Precondición ($P_{if}$)
		\item Poscondición ($Q_{if}$)
	    \end{itemize}
	    \item Hay que mostrar que si se cumple la precondición del if, se cumple la precondición valga la guarda o no ($P_{if} \Rightarrow Q_{if})$
	\end{itemize}	
	\item Demostración while
	\begin{itemize}
	    \item 
	    \textbf{while} ($B$) \{ \\
	    	//E1 \\
	    	cuerpo \\
	    	//E2 \\
	    \}
	    \item Componentes
	    \begin{itemize}
	    	\item Guarda (B)
		\item Precondición ($P_c$)
		\item Poscondición ($Q_c$)
		\item Invariante ($I$)
		\item Función variante ($V$)
		\item Cota ($c$)
	    \end{itemize}
	    \item Teorema del invariante: Sea un ciclo (como el de arriba, con estados E1,E2), si vale
	    \begin{itemize}
		\item $P_c \Rightarrow I$
		\item $(I \wedge \neg B) \Rightarrow Q_c$
		\item el invariante se preserva durante la ejecución del cuerpo
		\item $V$ es decreciente
		\item $(I \wedge V \leq c) \Rightarrow \neg B$
	    \end{itemize}
	    \subitem entonces para cualquier valor de las variables del programa que haga verdadera $P_c$, el ciclo termina y hace verdadera $Q_c$, es decir el ciclo es correcto respecto de su especificación
	\end{itemize}
    \end{itemize}
\end{document}
